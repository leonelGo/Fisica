\documentclass[11pt,letterpaper]{article}
\usepackage[utf8]{inputenc}
\usepackage[spanish]{babel}
\usepackage{amsmath}
\usepackage{amsfonts}
\usepackage{amssymb}
\usepackage[left=2cm,right=2cm,top=2cm,bottom=2cm]{geometry}
\usepackage{graphicx}
\usepackage{xcolor}
\usepackage[procnames]{listings}
\definecolor{keywords}{RGB}{255,0,90}
\definecolor{comments}{RGB}{0,0,113}
\definecolor{red}{RGB}{160,0,0}
\definecolor{green}{RGB}{0,150,0}
\definecolor{backcolour}{RGB}{220,220,220}
\usepackage{titling}
\usepackage{tcolorbox}
\usepackage{bigints}

\usepackage{pgf,tikz}
\usetikzlibrary{calc,patterns,decorations.pathmorphing,decorations.markings}
%\usetikzlibrary{backgrounds}
%\usetikzlibrary{decorations}

\usepackage{circuitikz}

\decimalpoint
 
\lstset{language=Python, 
        backgroundcolor=\color{backcolour},
        basicstyle=\ttfamily\small, 
        keywordstyle=\color{keywords},
        commentstyle=\color{comments},
        stringstyle=\color{red},
        showstringspaces=false,
        identifierstyle=\color{green},
        procnamekeys={def,class}}

\pagenumbering{gobble}

%%% AQUI INSERTE EL TÍTULO DE LA ACTIVIDAD Y SU NOMBRE %%%%%
\title{Actividad 2.2. Sistemas masa--resorte acoplados.}
\author{Inserte aquí su nombre.}
%%%%%%%%%%%%%%%%%%%%%%%%%%%%%%%%%%%%%%%%%%%%%%%%%%%%%%%%%%%%

\begin{document}
\hrule
\begin{center}
\noindent {\fontfamily{lmss}\selectfont
\huge{Simulación de Procesos Físicos}
}

\noindent {\fontfamily{lmss}\selectfont
\Large{\thetitle}
}

\noindent {\fontfamily{lmss}\selectfont
\Large{Alumno:  \theauthor}
}
\bigskip
\hrule
\end{center}

%%%%%%%%%%%%%%%%%%%%%%%%%%%%%%%%%%%%%%%%%%%%%%%%%%%%%%%%%%%%%

Dos objetos de masa $m_1$ y $m_2$ están acoplados mediante dos resortes con constantes $k_1$ y $k_2$ respectivamente; la punta izquierda del resorte de la izquierda está fijo a una pared, como se muestra en la figura.
\begin{figure}[h]
\centering
\ctikzset{mechanicals/scale=0.6}
\begin{circuitikz}[scale=0.6]

\fill[pattern=north east lines] (0,0) rectangle (0.25,2.5);
\draw (0.25,0) -- (0.25,2.5);

\draw (0.25,1.8) to[spring, l={$k_1$}] (2,1.8);
\draw (0.25,0.7) to[damper, l_=$b_1$] (2,0.7);
\draw[fill=gray!40] (4,0) rectangle (2,2.5);
\node at (3,1.25) {$m_1$};
\node at (3,-0.5) {$x_1$};

\draw (4,1.8) to[spring, l=$k_2$] (5.75,1.8);
\draw (4,0.7) to[damper, l_=$b_2$] (5.75,0.7);
\draw[fill=gray!40] (5.75,0) rectangle (7.75,2.5);
\node at (6.75,1.25) {$m_2$};

\node at (6.75,-0.5) {$x_2$};
\end{circuitikz}
\caption{Esquema de los sistemas masa--resorte acoplados.}
\end{figure}
Para agregar amortiguamiento al modelo añadimos los términos $b_1 x_1'$ y $b_2 x_2'$. El sistema de ecuaciones diferenciales acoplado de segundo orden para este sistema está dado por:
\begin{align*}
m_1 x_1'' + b_1 x_1' + k_1 x_1 - k_2 (x_2 - x_1) &= 0\\
m_2 x_2'' + b_2 x_2' + k_2 (x_2 - x_1) &= 0
\end{align*}


{\color{red}[Completar]}

\section*{Instrucciones}

\subsection*{Ejercicio 1.}

\begin{itemize}
\item Basándose en Fay y Graham (2003), añada los términos no lineales a las ecuaciones y al código.
\item Obtenga la solución numérica del ejemplo \textbf{3.1} del artículo.
\item Grafique la solución y realice la gráfica de fase para la solución de ambos sistemas (ver ejemplo de las gráficas en el artículo).
\item Discuta lo obtenido.
\end{itemize}

\subsection*{Ejercicio 2.}

\begin{itemize}
\item Basándose en Fay y Graham (2003), añada los términos de los forzamientos a las ecuaciones y al código.
\item Obtenga la solución numérica del ejemplo \textbf{4.1} del artículo.
\item Grafique la solución y realice la gráfica de fase para la solución de ambos sistemas (ver ejemplo de las gráficas en el artículo).
\item Discuta lo obtenido.
\end{itemize}
 
\newpage

%%%%%%%%%%%%%%%%%%%%%%%%%%%%%%%%%%%%%%%%%%%%%%%%%%%
\begin{tcolorbox}[colback=white]
\textbf{Código}.
\tcblower

\begin{lstlisting}

# C'odigo Fuente


\end{lstlisting}

\end{tcolorbox}

%%%%%%%%%%%%%%%%%%%%%%%%%%%%%%%%%%%%%%%%%%%%%%%%%%%

\begin{tcolorbox}[colback=white]
\textbf{Gráficas}.
\tcblower

\centering

\end{tcolorbox}

%%%%%%%%%%%%%%%%%%%%%%%%%%%%%%%%%%%%%%%%%%%%%%%%%%%%%
\begin{tcolorbox}[colback=white]
\textbf{Discusión}.
\tcblower


\end{tcolorbox}


\end{document}