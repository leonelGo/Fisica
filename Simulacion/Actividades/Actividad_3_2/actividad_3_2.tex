\documentclass[11pt,a4paper]{article}
\usepackage[utf8]{inputenc}
\usepackage[spanish,es-nodecimaldot]{babel}
\usepackage{amsmath}
\usepackage{amsfonts}
\usepackage{amssymb}
\usepackage[left=2cm,right=2cm,top=2cm,bottom=2cm]{geometry}
\usepackage{graphicx}
\usepackage{xcolor}
\usepackage[procnames]{listings}
\definecolor{keywords}{RGB}{255,0,90}
\definecolor{comments}{RGB}{0,0,113}
\definecolor{red}{RGB}{160,0,0}
\definecolor{green}{RGB}{0,150,0}
\definecolor{backcolour}{RGB}{220,220,220}
\usepackage{titling}
\usepackage{tcolorbox}
 
\lstset{language=Python, 
        backgroundcolor=\color{backcolour},
        basicstyle=\ttfamily\small, 
        keywordstyle=\color{keywords},
        commentstyle=\color{comments},
        stringstyle=\color{red},
        showstringspaces=false,
        identifierstyle=\color{green},
        procnamekeys={def,class}}

\pagenumbering{gobble}

%%% AQUI INSERTE EL TÍTULO DE LA ACTIVIDAD Y SU NOMBRE %%%%%
\title{Actividad 3.2. Conducción de calor en un medio heterogéneo.}
\author{Inserte aquí su nombre.}
%%%%%%%%%%%%%%%%%%%%%%%%%%%%%%%%%%%%%%%%%%%%%%%%%%%%%%%%%%%%

\begin{document}
\hrule
\begin{center}
\noindent {\fontfamily{lmss}\selectfont
\huge{Simulación de Procesos Físicos}
}

\noindent {\fontfamily{lmss}\selectfont
\Large{\thetitle}
}

\noindent {\fontfamily{lmss}\selectfont
\Large{Alumno:  \theauthor}
}
\bigskip
\hrule
\end{center}

%%%%%%%%%%%%%%%%%%%%%%%%%%%%%%%%%%%%%%%%%%%%%%%%%%%%%%%%%%%%%

Suponga que tenemos una barra de longitud $L$ construida con tres capas. Las fronteras entre las capas se denotan con $b_0 \cdots b_3$, donde $b_0 = 0$ y $b_3 = L$. Si las capas están hechas de materiales con propiedades diferentes y además estas se mantienen constantes dentro de la capa, podemos expresar el coeficiente de difusividad térmica como:
\begin{equation*}
\alpha(x)=\left\{\begin{array}{ll}
\alpha_0, & b_0 \leq x<b_1 \\
\alpha_1, & b_1 \leq x<b_2 \\
\alpha_2, & b_2 \leq x \leq b_3
\end{array}\right.
\end{equation*}
donde $\alpha_0 = 1$, $\alpha_1 = \alpha_0/2$ y $\alpha_2 = \alpha_1/2$.
Inicialmente se tiene una temperatura $T=0$ $^{\circ}$C a lo largo de toda la barra. En un tiempo $t=0$, se calienta el extremo izquierdo ($x=0$) manteniéndose a una temperatura de $T=100$ $^{\circ}$C y en el lado derecho se tiene un material aislante.

\section*{Instrucciones}

\begin{itemize}
\item Escriba un código para resolver la ecuación de difusión de calor para la barra compuesta de tres capas. Utilice $b1 = 0.25$ y $b_2 = 0.5$ para la posición de las fronteras entre capas. Se recomienda utilizar un método implícito.
\item Realice una gráfica para visualizar los resultados.
\item Discuta los resultados.
\end{itemize}

\newpage

%%%%%%%%%%%%%%%%%%%%%%%%%%%%%%%%%%%%%%%%%%%%%%%%%%%
\begin{tcolorbox}[colback=white]
\textbf{Código}.
\tcblower

\begin{lstlisting}

# C'odigo Fuente


\end{lstlisting}

\end{tcolorbox}

%%%%%%%%%%%%%%%%%%%%%%%%%%%%%%%%%%%%%%%%%%%%%%%%%%%

\begin{tcolorbox}[colback=white]
\textbf{Gráficas}.
\tcblower

\centering

\end{tcolorbox}

%%%%%%%%%%%%%%%%%%%%%%%%%%%%%%%%%%%%%%%%%%%%%%%%%%%%%
\begin{tcolorbox}[colback=white]
\textbf{Discusión}.
\tcblower


\end{tcolorbox}


\end{document}