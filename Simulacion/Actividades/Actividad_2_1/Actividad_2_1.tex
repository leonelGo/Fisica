\documentclass[11pt,letterpaper]{article}
\usepackage[utf8]{inputenc}
\usepackage[spanish]{babel}
\usepackage{amsmath}
\usepackage{amsfonts}
\usepackage{amssymb}
\usepackage[left=2cm,right=2cm,top=2cm,bottom=2cm]{geometry}
\usepackage{graphicx}
\usepackage{xcolor}
\usepackage[procnames]{listings}
\definecolor{keywords}{RGB}{255,0,90}
\definecolor{comments}{RGB}{0,0,113}
\definecolor{red}{RGB}{160,0,0}
\definecolor{green}{RGB}{0,150,0}
\definecolor{backcolour}{RGB}{220,220,220}
\usepackage{titling}
\usepackage{tcolorbox}
\usepackage{bigints}

\decimalpoint
 
\lstset{language=Python, 
        backgroundcolor=\color{backcolour},
        basicstyle=\ttfamily\small, 
        keywordstyle=\color{keywords},
        commentstyle=\color{comments},
        stringstyle=\color{red},
        showstringspaces=false,
        identifierstyle=\color{green},
        procnamekeys={def,class}}

\pagenumbering{gobble}

%%% AQUI INSERTE EL TÍTULO DE LA ACTIVIDAD Y SU NOMBRE %%%%%
\title{Actividad 2.1. Péndulo simple.}
\author{Inserte aquí su nombre.}
%%%%%%%%%%%%%%%%%%%%%%%%%%%%%%%%%%%%%%%%%%%%%%%%%%%%%%%%%%%%

\begin{document}
\hrule
\begin{center}
\noindent {\fontfamily{lmss}\selectfont
\huge{Simulación de Procesos Físicos}
}

\noindent {\fontfamily{lmss}\selectfont
\Large{\thetitle}
}

\noindent {\fontfamily{lmss}\selectfont
\Large{Alumno:  \theauthor}
}
\bigskip
\hrule
\end{center}
\bigskip{}

%%%%%%%%%%%%%%%%%%%%%%%%%%%%%%%%%%%%%%%%%%%%%%%%%%%%%%%%%%%%%
Considere un péndulo que consta de una varilla ligera de longitud $L$ a la que se une una bola de masa $m$. El otro extremo de la varilla está unido a una pared en un punto de modo que la bola del péndulo se mueva en un círculo centrado en este punto. La posición de la masa en el tiempo t se describe completamente por el ángulo $\theta(t)$ de la masa desde la posición recta hacia abajo y se mide en la dirección contraria a las manecillas del reloj.

Suponemos que las únicas dos fuerzas que actúan sobre el péndulo son la fuerza de gravedad y una fuerza debida a la fricción. La fuerza gravitacional es una fuerza constante igual a $ m g $ que actúa en dirección descendente; la componente de esta fuerza tangente al círculo de movimiento está dada por $ -m g \sin \theta $. 

Se considera que la fuerza debida a la fricción es proporcional a la velocidad, es decir, $ -b L \mathrm{d}\theta / \mathrm{d} t $ para una constante $ b> 0 $. Cuando no hay fuerza debido a la fricción $ (b = 0) $, tenemos un péndulo ideal cuyo movimiento está dado por:
\[ \theta ''(t) + \beta \theta '(t) + \alpha \sin(\theta(t)) = 0 \]
donde $\beta= b/m$ y $\alpha = g/L$, son constantes positivas. 

Podemos convertir la ecuación anterior en un sistema de ecuaciones de primer orden: 

{\color{red}[Completar]}

\section*{Instrucciones}

\begin{itemize}
\item Complete el desarrollo matemático para llegar a un sistema de ecuaciones de primer orden.
\item  Escriba un código en Python para resolver el sistema de ecuaciones que describen el movimiento del péndulo simple.
\item Utilice las condiciones iniciales $\theta(0) = \pi/2$ y $\omega(0) = 0$.
\item Obtenga la solución numérica para tres casos: 
\begin{enumerate}
\item Péndulo sin amortiguamiento.
\item Péndulo con amortiguamiento.
\item Péndulo con amortiguamiento crítico.
\end{enumerate}
\item Para cada caso obtenga el plano de fases y grafique sobre el mismo la solución numérica encontrada.
\item Complete con su código fuente y gráficas los campos indicados en la actividad.
\item Escriba las discusiones de sus resultados en la parte correspondiente del formato. Conteste: ¿que se observa al cambiar los valores de $b$?; ¿para que sirve el diagrama de fases? y ¿qué pasaría si se cambian las condiciones iniciales?.
\end{itemize}
 
\newpage

%%%%%%%%%%%%%%%%%%%%%%%%%%%%%%%%%%%%%%%%%%%%%%%%%%%
\begin{tcolorbox}[colback=white]
\textbf{Código}.
\tcblower

\begin{lstlisting}

# C'odigo Fuente


\end{lstlisting}

\end{tcolorbox}

%%%%%%%%%%%%%%%%%%%%%%%%%%%%%%%%%%%%%%%%%%%%%%%%%%%

\begin{tcolorbox}[colback=white]
\textbf{Gráficas}.
\tcblower

\centering

\end{tcolorbox}

%%%%%%%%%%%%%%%%%%%%%%%%%%%%%%%%%%%%%%%%%%%%%%%%%%%%%
\begin{tcolorbox}[colback=white]
\textbf{Discusión}.
\tcblower


\end{tcolorbox}


\end{document}